
% POSTER EXAMPLE
%
% This is an example of a relatively sane poster. The box structure (and the
% narrative in general) is what I would expect, but it is completely
% non-mandatory; you may include whatever you want. Preferably, erase the
% existing box structure after you read it, and start from scratch.
%
% The main communication requirements for the poster that should be satisfied
% are as such:
%
% - At the defense, it should help you talk for around 10 minutes about your
%   thesis, and convince the committee that you did something interesting and
%   sufficiently complicated. Prepare pictures that explain your main results.
%
% - It should quickly communicate the main idea of your thesis to a random
%   educated by-walker. Ideally, a moderately-witted MFF graduate who has never
%   heard about your thesis before should be able to get the main "rough idea"
%   in less than 1 minute by just looking at the poster.

% modify the fontscale parameter to make everything slighly bigger or smaller.
\documentclass[portrait,a0paper,fontscale=0.25]{baposter}

\usepackage[utf8]{inputenc}
\usepackage[T1]{fontenc}

% FONT CHOICES
% Posters do not need to be PDF/A; you can choose any relatable font from the
% TeX font catalogue without much risk. Sans-serif fonts are suggested for the
% posters; see https://tug.org/FontCatalogue/sansseriffonts.html
\usepackage[sfdefault]{Fira Sans}
%\usepackage[default]{droidsans}
%\usepackage[math]{iwona}
%\usepackage[defaultfam]{montserrat}
%\usepackage{cmbright}
%\usepackage{yfonts}\renewcommand{\familydefault}{\frakdefault}

\usepackage{color}
\usepackage{graphicx}
\usepackage{amssymb,amsmath}
\usepackage[export]{adjustbox} %allows using valign with \includegraphics

\renewcommand{\arraystretch}{1.5}

\usetikzlibrary{positioning}

% A WORD ABOUT COLORS
%
% This template is prepared with a relatively neutral gray background that
% gives decent box borders (with white and darker gray), does not clash with
% many colors (except for violet-brown and other mushroomish colors, perhaps)
% and gives a lot of space for highlighting stuff.
%
% Generally, other color variations are good too; there are no strict rules on
% the colors. Good choices include:
%
% - white backgrounds and differentiation of box headers by color (see
%   headerFontColor)
%
% - various slightly tinted backgrounds (try red!10 instead of black!3)
% 
% - dark backgrounds
%
% Keep in mind:
% - The normal "informative" text and figures should be DARK on LIGHT
%   background, not the other way around.
%
% - If you want a dark background, soften (darken) the box backgrounds a bit so
%   that they do not "shine" too much from the poster. Use \color{white} for
%   the heading, and switch the UK/MFF logos to white (see contents of logos/).
%
% - Do not mix too many color hues together. Most hues have their widely
%   accepted meaning (green: good result, red: problem, blue: information,
%   yellow: highlighter, brown: serious problem, violet: something really
%   weird/interesting/magic, depending on the shade).

\begin{document}

\color{black!80} % default font color
\begin{poster}{grid=false,
	eyecatcher=true,
	background=plain,
	bgColorOne=black!3, % background color
	columns=2,
	headerborder=none,
	textborder=none,
	headershape=rectangle,
	headershade=plain,
	boxshade=plain,
	boxColorOne=white,
	headershade=plain,
	headerColorOne=black!15, % box header background color
	headerFontColor=black,
	}%
	{\includegraphics[height=7em]{logos/mff-black.pdf}}
	{Extension of web-based interface for protein binding sites prediction}
	{\vspace{1ex} Lukáš Polák}
	{\includegraphics[height=7em]{logos/uk-red.pdf}}


%
% LEFT COLUMN
%

\begin{posterbox}[column=0,name=intro]{Introduction}

Detection of protein ligand-binding sites is a vital aspect of nowadays drug
discovery and development. Identification of the potential binding sites allows an
understanding of various molecule interactions, which is the first step of rational
drug discovery pipelines.

PrankWeb is a web-based tool developed at MFF UK for prediction of protein binding sites. It is based
on the P2Rank tool that uses machine learning to predict the binding sites.

%\begin{center}\begin{tikzpicture}[ultra thick, inner sep=1ex]
%\node[rectangle, rounded corners=1ex, draw=red!80!black, color=red!80!black, font=\huge\bfseries, rotate=21] {Problem!};
%\end{tikzpicture}\end{center}

\end{posterbox}

\begin{posterbox}[column=0, name=goals, below=intro, headerColorOne=cyan!60, boxColorOne=cyan!20]{Thesis goals}
There were two main goals of the thesis:
\begin{itemize}
\item Improve the web frontend, replace old and unsupported components with new ones.
\item Extend the server architecture to enable simple addition of modules for postprocessing of the predicted
binding sites.
\end{itemize}
\end{posterbox}

\begin{posterbox}[column=0, name=architecture, below=goals]{Server architecture}

This box will contain a description of the server architecture \dots

This box may contain an overview of the used methods, mathematics, program structure, etc.

Include a picture, because pictures are better. Thesis defense takes less than 10 minutes, no one can read a wall of text in that short time. (For comparison, the usual realistic poster visit on a conference takes 15 seconds, unless the poster manages to catch the attention in that short time.)

TODO: add a diagram of the server architecture

Lorem ipsum dolor sit amet, consectetur adipiscing elit. Sed euismod, nisl quis
consequat ultricies, nunc ipsum aliquam nunc, vitae aliquam nisl nunc eu
nunc. Nulla facilisi. Nulla facilisi. Nulla facilisi. Nulla facilisi.

Lorem ipsum dolor sit amet, consectetur adipiscing elit. Sed euismod, nisl quis
consequat ultricies, nunc ipsum aliquam nunc, vitae aliquam nisl nunc eu
nunc. Nulla facilisi. Nulla facilisi. Nulla facilisi. Nulla facilisi.

Lorem ipsum dolor sit amet, consectetur adipiscing elit. Sed euismod, nisl quis
consequat ultricies, nunc ipsum aliquam nunc, vitae aliquam nisl nunc eu
nunc. Nulla facilisi. Nulla facilisi. Nulla facilisi. Nulla facilisi.
Lorem ipsum dolor sit amet, consectetur adipiscing elit. Sed euismod, nisl quis
consequat ultricies.

%\tikzstyle{rec}=[rectangle, draw, rounded corners=1ex, font=\huge\bfseries]
%\begin{center}\begin{tikzpicture}[ultra thick, inner sep=1ex]
%\node[rec] (a) {Keep};
%\node[rec, circle, right=of a] (b) {it};
%\node[rec, right=of b] (c) {simple};
%\node[rec, densely dotted, below=6cm of c, font=\small] (notice) {\dots{}but precise!};
%\draw[->] (a) to (b);
%\draw[->] (b) to (c);
%\draw[dotted] (notice) to (c);
%\end{tikzpicture}\end{center}

\end{posterbox}

\begin{posterbox}[column=0, name=tech, below=architecture, headerColorOne=yellow!80!orange!95!black, boxColorOne=yellow!33]{Technologies}
\begin{itemize}
	\item \textbf{Frontend} - React (TypeScript), specialized open-source libraries Mol* and RCSB Saguaro 1D Viewer
	\item \textbf{Backend} - Python, Flask, Celery
\end{itemize}
The entire project is dockerized and deployed via Docker-compose. Some of the containers use different
technologies as well (e.g. Java), but those were not affected by the thesis.
\end{posterbox}

%
% FOOTER
%

\begin{posterbox}[column=0, span=2, name=footer, below=tech,
	textborder=none, headerborder=none, boxheaderheight=0pt,
	boxColorOne=black!3]{}
If some institute/grant/department sponsored the work, put an acknowledgement here.
\end{posterbox}

%
% RIGHT COLUMN
%
% It is usually best to fill most of the poster with your results and
% conclusions. Again, use simple annotated pictures wherever possible. Plots
% with measurements are perfect, tables are also good.
%

\begin{posterbox}[column=1, name=result1]{Frontend}
\begin{center}
	\includegraphics[width=0.8\linewidth]{img/Nový projekt (2).pdf}
\end{center}

One of the two replaced components is the 1D viewer. We employed the RCSB Saguaro 1D Viewer.
In the picture, the viewer shows the P06213 protein sequence with the predicted binding sites colored in the same theme as in Mol*.
The conservation and AlphaFold confidence scores are shown in the form of a histogram.

\begin{center}
	\includegraphics[width=0.8\linewidth]{img/new2.pdf}
\end{center}

In the second picture, the 3D representation of the protein 2SRC is shown.
As the 3D viewer, we used the Mol* library, which is actively developed.
The viewer is interactive and there are plenty of options to customize the visualization.
In this picture, the ball-and-stick representation is used for the protein, the binding sites are shown in the surface representation.
The atoms are colored by their conservation score.

\end{posterbox}

\begin{posterbox}[column=1, name=result2, below=result1]{Backend plugins}

As an example of the plugin system, we implemented a plugin that allows binding molecules into the predicted binding sites.
The plugin uses the AutoDock Vina software, which is a popular tool for molecular docking.
The implementation was integrated into the existing server architecture, so there were minimal changes to the existing codebase and deployment.

%TODO: add some pictures of the plugin or the example (docking)?
%\begin{center}
%\begin{tabular}{lrr}
% & \textbf{SomeProgram} & \textbf{ThisThesis} \\
%\hline
%Process A & 50\% & 58\% \\
%Time for A & 35 days & \textcolor{green!80!black}{35 seconds} \\
%Process B & \textcolor{red}{15\%} & 55\% \\
%Time for B & 1 day & 8 hours \\
%Price & 66.6 EUR & free
%\end{tabular}
%\end{center}
\end{posterbox}

%\begin{posterbox}[column=1, name=result3, below=result2, headerColorOne=green!50!yellow, boxColorOne=green!10]{Main result}
%\large\bfseries
%\vspace{1ex}
%\begin{center}
%Program ThesisProgram solves the problem better than OtherProgram if X, and faster if Y.
%\end{center}
%\vspace{.5ex}
%\end{posterbox}

\begin{posterbox}[column=1, name=conclusion, below=result2, bottomaligned=tech]{Source codes \& Contact}

%TODO: maybe add a QR code with the link to the repository or the running instance.

\begin{minipage}[t]{\linewidth}
	\begin{minipage}[t]{0.75\linewidth}
		Thesis supervisor: \textbf{doc. RNDr. David Hoksza, Ph.D.}, Department of Software Engineering\\

		GitHub repos: \textbf{github.com/cusbg/prankweb}, \textbf{github.com/cusbg/p2rank-framework}\\

		Running instance: \textbf{prankweb.cz}
		%TODO: use the cusbg or mine?
	\end{minipage}
	\quad
	\begin{minipage}[t]{0.20\linewidth}
		\vspace{-2ex}
		\includegraphics[width=1\linewidth]{./img/dbe7ed14a5bd51a8518c9a1d66eaa759.png} %TODO: use running instance or github?
		\begin{center}
			Try it out yourself!
		\end{center}
	\end{minipage}
\end{minipage}


\end{posterbox}

\end{poster}
\end{document}
